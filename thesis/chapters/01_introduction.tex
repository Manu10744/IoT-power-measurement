% !TeX root = ../main.tex
% Add the above to each chapter to make compiling the PDF easier in some editors.

\chapter{Introduction}\label{chapter:introduction}
IoT applications are continuously growing in importance in a multitude of fields such as traffic management, healthcare, agriculture and home automation. As IoT-enabled devices are usually equipped with a variety of sensors, they continuously produce data in different shapes and sizes, which has to be delivered to connected devices using heterogeneous communication patterns. Consequently, this can result in an enormous energy consumption. Due to the fact that the lifetime of batteries is limited, an efficient power management is key to maximize their expected lifetime and thus a highly important task that must not be neglected. Usually, though, this is no factor that is taken care of by the developers of an IoT application as they generally focus on implementing the logics rather than taking secondary aspects into account like device failures or the amount of energy consumed.\\  
Additionally, the operation of IoT devices and corresponding applications introduces significant challenges and requirements including the need for high availability and scalability, a wide range of computing and storage resources, sufficient network capabilities and fault tolerance. To this end, innovative IoT platforms have been developed which are built on top of two modern concepts originating from the area of cloud computing, namely serverless computing and Function-as-a-Service (FaaS). While the former helps increase the productivity by taking care of the server and hardware configuration and management on behalf of the user, the latter provides an event-driven architecture that facilitates the application development and decreases the complexity by breaking the application up into stateless functions designed to run as standalone units, typically inside a containerized runtime environment. Once serverless functions have been deployed on the respective FaaS platform, they can be invoked directly by simple HTTP requests or alternatively by a wide range of certain types of events, that are often referred to as triggers. As a consequence, developers and business operators benefit from being relieved the tremendous burden of having to setup and manage the deployment environment themselves while the actual software can be operated using a highly flexible, decentralized infrastructure utilizable at a remarkable cost-efficiency as its usage is typically charged based on a pay-as-you-use payment model.\\
Though, these platforms are limited to homogeneous clusters of nodes as well as to homogeneous functions and data access behavior of functions during scheduling is not taken into account, which are issues tackled by the recently proposed Function Delivery Network (FDN). Essentially, the FDN is a network of distributed, heterogeneous target platforms that supports heterogeneous clusters and heterogeneous functions requiring different amounts of computational and data resources. This is achieved by introducing the concept of Function-Delivery-as-a-Service (FDaaS), which takes care of delivering a function to the appropriate target platform - representing a cluster composed of homogeneous nodes and a FaaS platform at the top level - based on its individual resource demands~\parencite{fdn}. \\
Presently, edge devices like mini-computers can already be integrated into a serverless IoT platform, which allows for the deployment of serverless functions to the edge systems.~\parencite{fdn}. Though, measuring the energy consumption of functions executed on edge devices is a non-trivial task, since it depends on a variety of factors including computational effort, memory consumption and available hardware resources, which are typically quite different in that regard. Additionally, the scenario of multiple functions running simultaneously on the same edge device has to be taken into account as well. Therefore, this thesis aims on providing an answer on the following questions.

\begin{enumerate}
  \item Given an existing monitoring infrastructure, how can a power measurement for edge systems be implemented and integrated? In this regard, the Function Delivery Network which provides VMs, HPC servers and edge computers will be used in combination with Grafana as the monitoring platform.
  \item How can the energy consumption of serverless functions delivered to the edge systems be put in relation to the total power consumption and to what extent can it be influenced by utilizing different hardware resources or software?
  \item What could be an appropriate energy model that can be employed by the Function Delivery Network in order to efficiently schedule functions?
\end{enumerate}

A Natural Language Processing application designed as a serverless function is chosen to serve as the basis for this work. The power measurements will be carried out by an ESP32 powermeter which has been developed by the Chair of Computer Architecture \& Parallel Systems at the Technical University of Munich. By providing an appropriate interface, the edge computers will be enabled to connect to the powermeter via WiFi in order to retrieve the measurement data, which will be subsequently pushed to the database and Grafana.
\\
\\
The rest of this thesis is structured as follows. Section 2 gives a brief overview of the employed cluster monitoring tools \textit{Prometheus} and \textit{Grafana} and introduces the different types of edge computers that are going to be involved in this work. Additionally, a more detailed explanation of the Function Delivery Network and its components is given. The experimental design and associated infrastructure is explained in Section 3. In Section 4, ....